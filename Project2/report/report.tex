\documentclass[11pt]{article}
\usepackage{acl2014}
\usepackage{times}
\usepackage{url}
\usepackage{graphicx}
%\usepackage{latexsym}


%figures:
\usepackage{tikz}
\usetikzlibrary{trees,positioning,backgrounds}
%\usepackage{tikz-qtree}
\usepackage{subcaption}

%references and keeping floats in place
\usepackage[pdftex,pdfborder={0 0 0},unicode,breaklinks,hyperfootnotes=false,bookmarks]{hyperref}
\usepackage[section]{placeins}

\usepackage{amsmath} 
\renewcommand{\vec}[1]{\mathbf{#1}}

\title{Statistical Structure in Language Processing \\ Phrase based models}
\author{ Cristina G\^arbacea\\
  10407936 \\
  {\small \tt cr1st1na.garbacea@gmail.com} 
  \\\And
  Sara Veldhoen \\
10545298   \\
  {\small \tt sara.veldhoen@student.uva.nl} \\}

\date{}

\begin{document}

\maketitle

\begin{abstract}
Bla bla bla
\end{abstract}

\section{Introduction}



\section{Phrase Extraction and weight estimation}
\label{phraseExtraction}

In this section, we present our approach to the extraction of phrase pairs from the corpus. Subsequently, we 

\paragraph{Phrase Extraction}
The number of possible phrase pairs per sentence pair is huge: each sentence can be partitioned in a vast amount of ways, and each partition could form a phrase pair with any partition in the paired sentence. 

In order to reduce the space, we consider only phrase pairs that are consistent with the alignments produced by IBM models.
As in \cite{Koehn:2010}, consistency is defined as follows:
\begin{align*}
\langle \bar{e},\bar{f}\rangle\text{ is consistent with }A &\Leftrightarrow\\
\forall e_i\in \bar{e}: &\langle e_i,f_j\rangle \in A \Rightarrow f_j \in \bar{f}, \\
\text{ and }\forall f_j\in \bar{f}: &\langle e_i,f_j\rangle \in A \Rightarrow e_i \in \bar{e}, \\
\text{ and }\exists e_i \in \bar{e}, f_j\in \bar{f}: &\langle e_i,f_j\rangle \in A.\\
\end{align*}

For this assignment, the symmetrized alignments of the corpus sentences were given. 
We base our extraction algorithm on the one presented in \cite[page 133]{Koehn:2010}.
We iterate over all windows up to a certain length in the English sentence, and find the foreign windows that are consistent given the alignment. For all valid pairs of windows, we extract the corresponding phrase pair.

\paragraph{Conditional Probability Estimates}

\paragraph{Joint Probability Estimates}
In \cite{marcu2002} quite a different approach is taken to phrase based translation. The idea of a noisy channel, that a foreign sentence is a corrupted version of an original English sentence, is abandoned. Rather, the two sentences are considered different substantiation of a bag of concepts. In this framework, the probability of a phrase pair is a joint probability conditioned on a concept. In practice, we do not explicitly model the concept but view the phrase pair itself as a concept, so its weight is just the joint translation probability of the two phrases: $t(\bar{e},\bar{f}$.

The estimation of the translation probabilities in \cite{marcu} is done in an adapted version of expectation maximization. In the first step, high-frequency n-grams are determined in the bilingual corpus 



\section{Experiments and Results}
\label{Eval}

\section{Conclusion}
\label{Concl}

\begin{thebibliography}{}
%\bibitem[\protect\citename{Gusfield}1997]{Gusfield:97}
%Dan Gusfield.
%\newblock 1997.
%\newblock {\em Algorithms on Strings, Trees and Sequences}.
%\newblock Cambridge University Press, Cambridge, UK.

\bibitem[1]{Koehn:2010}
Philipp Koehn,
\newblock 2010.
\newblock {\em Statistical Machine Translation}.
\newblock Cambridge University Press.

\bibitem[2]{marcu2002}
{Daniel Marcu and William Wong},
\newblock 2002.
\newblock {\em A phrase-based, joint probability model for statistical machine translation}.
\newblock {Association for Computational Linguistics}.
%@inproceedings{marcu2002phrase,
%  title={A phrase-based, joint probability model for statistical machine translation},
%  author={Marcu, Daniel and Wong, William},
%  booktitle={Proceedings of the ACL-02 conference on Empirical methods in natural language processing-Volume 10},
%  pages={133--139},
%  year={2002},
%  organization={Association for Computational Linguistics}
%}

\end{thebibliography}

\end{document}
