\documentclass[11pt]{article}
\usepackage{acl2014}
\usepackage{times}
\usepackage{url}
\usepackage{graphicx}
%\usepackage{latexsym}


%figures:
\usepackage{tikz}
\usetikzlibrary{trees,positioning,backgrounds}
%\usepackage{tikz-qtree}
\usepackage{subcaption}

%references and keeping floats in place
\usepackage[pdftex,pdfborder={0 0 0},unicode,breaklinks,hyperfootnotes=false,bookmarks]{hyperref}
\usepackage[section]{placeins}

\usepackage{amsmath} 
\renewcommand{\vec}[1]{\mathbf{#1}}

\title{Statistical Structure in Language Processing \\Using multi-parallel data for phrase-table improvement}
\author{ Cristina G\^arbacea\\
  10407936 \\
  {\small \tt cr1st1na.garbacea@gmail.com} 
  \\\And
  Sara Veldhoen \\
10545298   \\
  {\small \tt sara.veldhoen@student.uva.nl} \\}

\date{}

\begin{document}

\maketitle

\begin{abstract}
In this paper we continue to extend upon our previous assignment by building a phrase pair extraction tool that would use evidence from aligned parallel corpora. Unlike in our previous work where we would only use parallel corpora consisting of two languages, this time we use evidence from multilingual parallel aligned corpora made available by Europarl. We investigate how word alignments between several languages can be of use in extracting consistent phrase pairs. %along with their conditional and joint probabilities 
%and consider how word alignment symmetrization can be improved from multilingual word alignments of the same target sentence. 
We present an algorithm to do so, and experimented with different symmetrization heuristics to be used as an input. Furthermore, we investigated how the choice of reference languages influences improvement. We compare our results to a baseline system, and to a different approach to phrase table filtering, based on significance testing.
\end{abstract}

\section{Introduction}
Phrase-based statistical machine translation relies on estimates of translation probabilities for pairs of phrases, stored in a so-called phrase table.
In principle, the set of possible phrase pairs is far too huge to be computationally manageable, especially if the size of the corpus grows.
Therefore, we need to restrict this set in some way, to keep only useful/ meaningful phrase pairs. This reduction might be the main challenge in phrase based machine translation.

In the approach we took in the previous assignment, the reduction was based on Giza++ output: only those phrase pairs that were consistent with symmetrized word alignments were added to the phrase table. This is a standard approach, that yields a large reduction and proves useful for translation, obtaining for instance a BLEU score of 24.68 in our previous assignment \cite{previous}.
%But the MT community has moved forward, and can do better now.

The possibility that we aim to investigate in this project, is to use multi-parallel corpora. These data consist of documents in more than two languages with aligned sentences. The process of incorporating evidence from multilingual data in a single system is called \emph{triangulation}, and presents the advantage of using a wider range of parallel corpora for training. 

\section{Related work}

%Chen, Eisele and Kay 
Two methods are presented in \cite{chen} to filter the phrase table for a language pair% which we will dubb \emph{source-target}
, based on an intermediate third \emph{bridge} language. 
 Both methods assume an existing {\em source-target} phrase table, based on Giza++, and filter its entries with evidence from phrase tables {\em source-bridge} and {\em bridge-target}.

In method 1, for each phrase pair $\langle s, t\rangle \in source-target$, it is kept if there is an entire phrase $b$ in the bridge language such that $\langle s,b\rangle \in source-bridge$ and also $\langle b,t\rangle \in bridge-target$.

Method 2 is somewhat more lenient, in that it looks at the words occurring in the phrases instead of an exact match of the entire phrase. An overlap score is assigned to each phrase pair, based on the intersection of the vocabularies in the phrases. The filtering is done by placing a threshold on this score.

% Cohn and Lapata
The authors of \cite{cohn} focus less on reducing the phrase tables, but use triangulation to obtain high quality phrase tables from multilingual data, even if they are not from the same corpus. They use a summation over several intermediate languages to form a probability estimate: $p(s|t)=\sum_i p(s|i)p(i|t)$. Interestingly, the triangulated phrase table is trained separate from the standard phrase table, so that it can be used as a distinct feature in decoding.

%Johnson
In \cite{Johnson} the bulk of phrase table is reduced based on the significance testing of phrase pair co-occurence in the parallel corpus. The authors present two significance testing methods, namely Chi-squared and Fisher exact test, and experiment with different pruning threshold values ranging from 14 to 25. They outline that while the savings in terms of number of phrases discarded are considerable, the translation quality as measured by the BLEU score is preserved, and even more surprisingly it can even increase, making their approach a valuable contribution to the field of machine translation.


%\section{Goals}
%In this project we are planning to build a phrase pair extraction tool for phrases of up to length 7 that would use evidence from multilingual aligned corpora. To achieve our goal we aim to use Europarl % cite EUroparl
%data for Dutch, English, Romanian, French, German. % maybe add some more?
%We are planning to investigate how we can use the evidence from word alignments between several languages to extract, rather than filter, phrase pairs and estimate their conditional and joint probabilities. For this, we consider the possibility to improve word alignment symmetrization from multilingual word alignments of the same target sentence.



\section{Word-alignments}
Phrase pair extraction is based on symmetrized word alignments. 

The bidirectional word alignments are obtained with {\tt Giza++}, which is a combination of IBM-models. Although the quality of these alignments is not very high on itself, they have proven to be a useful guide in the extraction of phrase pairs. 

%The symmetrization is done by Moses, which provides several heuristics to do so. We use the default setting for our baseline, i.e. grow-diag-final-and. 

 Because of the design of IBM models, the Giza word-alignments are one to many. For symmetrized word alignment, where many-to-many alignments are possible, Giza alignments for both directions are used and their alignment points combined. Each word-alignment can be viewed as a matrix with words in both languages along the axes, and binary values in the cells: either the words are aligned, or not. If we combine the two directions, we introduce new values for the cells: the alignment point is either present in both directions ($\mathcal B$), in the source-target alignment ($\mathcal S$), in the target-source alignment ($\mathcal T$), or in none (empty).

The Moses alignment symmetrization is aimed to recast such a matrix back into a binary matrix. Union and intersection are the extreme choices, several  heuristics are available in Moses that compromise them:\begin{itemize}
\item {\tt union} contains all non-empty cells. This heuristic has a high recall, but might have many false positives.
\item  {\tt intersection} keeps only the $\mathcal B$ cells. These are the high-confidence points, thus improving precision but (generally) dropping recall.
\item {\tt grow} starts from the intersection and adds block-neighboring non-empty cells.
\item {\tt grow-diag} is like {\tt grow}, but it also includes diagonally neighboring non-empty cells.
\item {\tt final} is used after either heuristic, to add as many as possible unaligned words that are not neighbored.  %Don't fully understand how though. This intruduces the resumed-unterbrochene alignment in the example
%\item {\tt grow-diag-final-and} is not explained in the Moses manual
\end{itemize}

In these heuristics, Moses does not distinguish between $\mathcal S$ and $\mathcal T$ cells. Since we use multi-parallel data that all translates to the same target (English), our extraction is not entirely symmetrical and this distinction might be of importance. 

Our base-line 

The symmetrized alignments are used for reducing the space of phrase pairs. That means that having less alignment points, as in the intersection heuristic, is actually the more lenient choice for the phrase extraction step: you do not discard phrase pairs based on weak evidence. 



\section{Experiments}

\subsection{Data}

We use the Europarl version 7 parallel data for English to respectively French, German, Danish, Italian, Dutch, Spanish, and Portuguese. This choice was based on the fact that parallel data exists for parliament proceedings of the same time interval for these languages, so that they are in a sense parallel beyond language level.
We excluded Greek because of the following remark: ``Some recent Greek data has only parts of transcripts in the files." Moreover, the Greek data was not in a readable format, because of the deviant alphabet.

We use a script that comes with the Moses installation for initial preprocessing. It discards empty lines and lines with sentences that exceed a length threshold. Furthermore, all text is lowercased.

The Europarl recommends to set aside the Q4/2000 portion of the data for testing. However, the clean parallel corpora don't have time annotation, so there is no easy way to make this split. Therefore we decided to just create a split ourselves. Each 50th sentence is removed from the corpus and added to a separate test file. The ratio is based on the fact that the Q4/2000 portion would be 3/(16*12-4)=0.016 of the data.


\paragraph{Sentence alignment}
The parallel data is sentence aligned, which means that each line in the source text corresponds to a line in the target text.
In order for our phrase extraction to work, we want one single English (target) text file to be aligned to all languages.

Our implementation is quite ad hoc, in that it compares the English sides and just discards those lines that are different. This means we discard part of the data, because in the original parallel data sometimes multiple sentences are concatenated in the sentence alignment. This approach could probably be refined to lose less data. 

For a given English file that we pick randomly, we keep only those sentences that are present in the other English files as well, with the aim to identify common sentences in all English files from the parallel corpora. For each of these sentences we preserve the index of the occurrence so that we can retrieve the corresponding translations from the foreign files found at the very same position. For computation efficiency reasons, instead of looping though all the lines in the foreign file, which would take a lot of time, we prefer to keep track of the last retrieved position $i_l$ from the English file and count the number of lines $n_s$ skipped since then to search for the corresponding foreign phrase inside a window defined as $i_l+n_s*3$. We chose 3 as a safe margin leaving from the premises that we include all the sentences in the neighborhood of the expected index. Thus we report time and memory gains in the preprocessing step of our corpus.
%I think our algorithm is still not very efficient, so I wouldn't mention time and memory gains explicitly


\subsection{}

We run Moses steps 1 and 2 with default setting on all language pairs, to obtain Giza word alignments. 


\section{Results}
%Settings:
% Pool of languages - large (7 languages), small-same (3 roman languages: French, Italian, Spanish), small-different (3 very different languages: Greek, Danish, Portoguese)

%Idea: only use pruned data for extraction - prevent sparsity

%Relative size of the data after preprocessing
%Size of the phrase tables for different settings
%Decoding performance


\section{Conclusion}

% Future: apllicable for parallel data from different domains, so not language-parallel (is that possible?)
A major drawback of our approach is the constraint that the multi-parallel corpus has to be aligned, so that the target-sides coincide. Apart from the possibility to improve the tool for extracting such data from a corpus like Europarl, it may be interesting to investigate the possibility to abandon this constraint at all. For instance, by using an existing (bilingually trained) system to provide translations for the English sentences in other languages. This would however severely aggravate the training of the model, so the question is whether the reduction in phrase table expected from our method is worth it.
%We can say more about thit once we actually have results..


\begin{thebibliography}{}

\bibitem[1]{previous}
{Cristina G{\^a}rbacea and Sara Veldhoen},
\newblock{2014}.
\newblock{\em Phrase based models},
\newblock Statistical Structure in Language Processing assignment

\bibitem[2]{chen}
{Yu Chen, Andreas Eisele, Martin Kay},
\newblock{2008}.
\newblock{\em Improving Statistical Machine Translation Efficiency by Triangulation},
\newblock LREC

\bibitem[3]{cohn}
{Trevor Cohn and Mirella Lapata},
\newblock{2007}.
\newblock{\em Machine translation by triangulation: Making effective use of multi-parallel corpora},
\newblock {ANNUAL MEETING-ASSOCIATION FOR COMPUTATIONAL LINGUISTICS}

\bibitem[4]{Johnson}
{J Howard Johnson, Joel Martin, George Foster and Roland Kuhn},
\newblock{2007}.
\newblock{\em Improving Translation Quality by Discarding Most of the Phrasetable},
\newblock {In Proceedings of EMNLP-CoNLL}

%%\bibitem[\protect\citename{Gusfield}1997]{Gusfield:97}
%%Dan Gusfield.
%%\newblock 1997.
%%\newblock {\em Algorithms on Strings, Trees and Sequences}.
%%\newblock Cambridge University Press, Cambridge, UK.
%
%\bibitem[1]{Koehn:2010}
%Philipp Koehn,
%\newblock 2010.
%\newblock {\em Statistical Machine Translation}.
%\newblock Cambridge University Press.
%
%\bibitem[2]{marcu2002}
%{Daniel Marcu and William Wong},
%\newblock 2002.
%\newblock {\em A phrase-based, joint probability model for statistical machine translation}.
%\newblock {Association for Computational Linguistics}.
%%@inproceedings{marcu2002phrase,
%%  title={A phrase-based, joint probability model for statistical machine translation},
%%  author={Marcu, Daniel and Wong, William},
%%  booktitle={Proceedings of the ACL-02 conference on Empirical methods in natural language processing-Volume 10},
%%  pages={133--139},
%%  year={2002},
%%  organization={Association for Computational Linguistics}
%%}
%
%\bibitem[3]{och1999}
%{Franz Josef Och, Christoph Tillmann, and Hermann Ney},
%\newblock 1999.
%\newblock {\em Improved alignment models for statistical machine translation}
%\newblock {Proceedings of the Joint SIGDAT Conf. on Empirical Methods in Natural Language Processing and Very Large Corpora}.
%
%\bibitem[4]{mosesurl}
%{Koehn, Philipp},
%\newblock 2014.
%\newblock {\em Statistical Machine Translation System. User Manual and Code Guide}
%\newblock {\url{http://statmt.org/moses/manual/manual.pdf}}.
%
%\bibitem[5]{moses}
%{Koehn, Philipp, et al}
%\newblock{2007}
%\newblock{\em Moses: Open source toolkit for statistical machine translation}.
%\newblock{Proceedings of the 45th Annual Meeting of the ACL on Interactive Poster and Demonstration Sessions}.
%
%\bibitem[6]{giza++}
%{Franz Josef Och and Hermann Ney}
%\newblock{2003}
%\newblock{\em A Systematic Comparison of Various Statistical Alignment Models}.
%\newblock{Computational Linguistics, 1:29, 19-51}.
%
%\bibitem[7]{srilm}
%{Andreas Stolcke}
%\newblock{2002}
%\newblock{\em SRILM - An Extensible Language Modeling Toolkit}.
%\newblock{901--904}.
%
%\bibitem[8]{bleu}
%{Kishore Papineni and Salim Roukos and Todd Ward and Wei-Jing Zhu}
%\newblock{2002}
%\newblock{\em BLEU - A Method for Automatic Evaluation of Machine Translation}.
%\newblock{Proceedings of the 40th Annual Meeting of the Association for ACL}.
%\newblock{311--318}.


\end{thebibliography}

\end{document}
