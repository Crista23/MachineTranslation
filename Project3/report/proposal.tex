\documentclass[11pt]{article}
\usepackage{acl2014}
\usepackage{times}
\usepackage{url}
\usepackage{graphicx}
%\usepackage{latexsym}


%figures:
\usepackage{tikz}
\usetikzlibrary{trees,positioning,backgrounds}
%\usepackage{tikz-qtree}
\usepackage{subcaption}

%references and keeping floats in place
\usepackage[pdftex,pdfborder={0 0 0},unicode,breaklinks,hyperfootnotes=false,bookmarks]{hyperref}
\usepackage[section]{placeins}

\usepackage{amsmath} 
\renewcommand{\vec}[1]{\mathbf{#1}}

\title{Statistical Structure in Language Processing \\Project 3 research proposal}
\author{ Cristina G\^arbacea\\
  10407936 \\
  {\small \tt cr1st1na.garbacea@gmail.com} 
  \\\And
  Sara Veldhoen \\
10545298   \\
  {\small \tt sara.veldhoen@student.uva.nl} \\}

\date{}

\begin{document}

\maketitle


\section{Introduction}
This proposal continues to build on the phrase based machine translation approach.
In principle, the set of possible phrase pairs is far too huge to be computationally manageable. 
Especially if the size of the corpus grows, the 
Therefore, we need to restrict this set in some way, to keep only useful/ meaningful phrase pairs. This reduction might be the main challenge in phrase based machine translation.

In the baseline approach we took in the previous assignment, this reduction was based on Giza++ output: only those phrase pairs that were consistent with symmetrized word alignments were added to the phrase table.
This yields a large reduction and proves useful for translation, obtaining for instance a BLEU score of 24.68 in our previous assignment \cite{previous}.
But the MT community has moved forward, and can do better now.

One possibility that we aim to investigate in this project, is to use multi-parallel corpora. These data consist of documents in more than two languages with aligned sentences. The process of incorporating evidence from multilingual data in a single system is called \emph{triangulation}. 

\section{Related work}

%Chen, Eisele and Kay 
Two methods are presented in In \cite{chen} to filter the phrase table for a language pair which we will dubb \emph{source-target}, based on an intermediate third \emph{bridge} language. 
 Both methods assume an existing {\em source-target} phrase table, based on Giza++, and filter its entries with evidence from phrase tables {\em source-bridge} and {\em bridge-target} where {\em bridge} is an intermediate language.

In method 1, for each phrase pair $\langle s, t\rangle \in source-target$, it is kept if there is an entire phrase $b$ in the bridge language such that $\langle s,b\rangle \in source-bridge$ and also $\langle b,t\rangle \in bridge-target$.

Method 2 is somewhat more lenient, in that it looks at the words occurring in the phrases instead of an exact match of the entire phrase. An overlap score is assigned to each phrase pair, based on the intersection of the vocabularies in each phrase. The filtering is done by placing a threshold on this score.

% Cohn and Lapata



\begin{thebibliography}{}

\bibitem[1]{previous}
{Cristina G{\^a}rbacea and Sara Veldhoen},
\newblock{2014}.
\newblock{\em Phrase based models},
\newblock Statistical Structure in Language Processing assignment

\bibitem[2]{chen}
{Yu Chen, Andreas Eisele, Martin Kay},
\newblock{2008}.
\newblock{\em Improving Statistical Machine Translation Efficiency by Triangulation},
\newblock LREC
%%\bibitem[\protect\citename{Gusfield}1997]{Gusfield:97}
%%Dan Gusfield.
%%\newblock 1997.
%%\newblock {\em Algorithms on Strings, Trees and Sequences}.
%%\newblock Cambridge University Press, Cambridge, UK.
%
%\bibitem[1]{Koehn:2010}
%Philipp Koehn,
%\newblock 2010.
%\newblock {\em Statistical Machine Translation}.
%\newblock Cambridge University Press.
%
%\bibitem[2]{marcu2002}
%{Daniel Marcu and William Wong},
%\newblock 2002.
%\newblock {\em A phrase-based, joint probability model for statistical machine translation}.
%\newblock {Association for Computational Linguistics}.
%%@inproceedings{marcu2002phrase,
%%  title={A phrase-based, joint probability model for statistical machine translation},
%%  author={Marcu, Daniel and Wong, William},
%%  booktitle={Proceedings of the ACL-02 conference on Empirical methods in natural language processing-Volume 10},
%%  pages={133--139},
%%  year={2002},
%%  organization={Association for Computational Linguistics}
%%}
%
%\bibitem[3]{och1999}
%{Franz Josef Och, Christoph Tillmann, and Hermann Ney},
%\newblock 1999.
%\newblock {\em Improved alignment models for statistical machine translation}
%\newblock {Proceedings of the Joint SIGDAT Conf. on Empirical Methods in Natural Language Processing and Very Large Corpora}.
%
%\bibitem[4]{mosesurl}
%{Koehn, Philipp},
%\newblock 2014.
%\newblock {\em Statistical Machine Translation System. User Manual and Code Guide}
%\newblock {\url{http://statmt.org/moses/manual/manual.pdf}}.
%
%\bibitem[5]{moses}
%{Koehn, Philipp, et al}
%\newblock{2007}
%\newblock{\em Moses: Open source toolkit for statistical machine translation}.
%\newblock{Proceedings of the 45th Annual Meeting of the ACL on Interactive Poster and Demonstration Sessions}.
%
%\bibitem[6]{giza++}
%{Franz Josef Och and Hermann Ney}
%\newblock{2003}
%\newblock{\em A Systematic Comparison of Various Statistical Alignment Models}.
%\newblock{Computational Linguistics, 1:29, 19-51}.
%
%\bibitem[7]{srilm}
%{Andreas Stolcke}
%\newblock{2002}
%\newblock{\em SRILM - An Extensible Language Modeling Toolkit}.
%\newblock{901--904}.
%
%\bibitem[8]{bleu}
%{Kishore Papineni and Salim Roukos and Todd Ward and Wei-Jing Zhu}
%\newblock{2002}
%\newblock{\em BLEU - A Method for Automatic Evaluation of Machine Translation}.
%\newblock{Proceedings of the 40th Annual Meeting of the Association for ACL}.
%\newblock{311--318}.


\end{thebibliography}

\end{document}
